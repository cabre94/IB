\section*{Ejercicio 1 - Modelo de Hopfield sin ruido}

Para este este ejercicio, se considero una red de Hopfield sin ruido para solucionar el problema de memorizar patrones. Este problema consiste en memorizar un conjunto de patrones $\vec{\xi}^{\mu}$ de manera que al presentarle un nuevo patrón, la red retornara el patrón almacenado mas cercano a la entrada.

Se comenzó generando $p$ patrones $\xi_{i}^{\mu}$ con $i=1,\dots,N$ y $\mu=1,\dots,p$, en donde cada uno de los valores es $\pm1$ con igual probabilidad.

Luego se evaluó la matriz de conexiones $J$ de la red según

\begin{equation}
    J_{ij} =
    \left\{ \begin{array}{lcc}
        \frac{1}{N} \sum_{\mu=1}^{p} \xi_{i}^{\mu} \xi_{j}^{\mu}  & \text{si } i\neq j \\
        0           & \text{si } i=j. \\
        \end{array}
    \right.
    \label{eq:J}
\end{equation}

Fijados los valores de $N$ y $p$, se iteró la dinámica determinista dada por 
\begin{equation}
    S_{i} = \text{sgn} \left( \sum_{j=1}^{N} J_{ij} S_{j} \right)
\end{equation}
tomando como condición inicial $\vec{S} = \vec{\xi}^{\mu}$ y recorriendo el vector $\vec{S}$ de manera aleatoria. La dinámica finaliza una vez alcanzado un punto fijo $\vec{S}^{\mu}$ y luego se calculo el \textit{overlap} con el patrón
\begin{equation}
    m^{\mu} = \frac{1}{N}\sum_{i=1}^{N} S_{i}^{\mu} \xi_{i}^{\mu}.
\end{equation}
Este proceso se repitió para cada uno de los $p$ patrones.

\begin{figure}[htb!]
    \centering
    \includegraphics[width=\textwidth]{../1.pdf}
    \caption{Histogramas del \textit{overlap} para distintos tamaños $N$ de la red de Hopfield y parámetros de carga $\alpha$.}
    \label{fig:01}
\end{figure}

En la Fig. \ref{fig:01} se observan distintos histogramas de los \textit{overlaps} obtenidos para distintos valores de $N$ y del parámetro de carga $\alpha = \dfrac{p}{N}$. Para distintos tamaños y para $\alpha=0.12$ se observa que la mayor parte de los \textit{overlaps} es cercano a $1$, lo cual indica que el sistema puede reconocer los patrones que se almacenaron en la matriz de conexiones $J$. En el limite $N\rightarrow\infty$, los valores de \textit{overlap} deberían ser tales que $m\approx1$ para $\alpha < \alpha_{c} \approx 0.138$ y $m=0$ para $\alpha > \alpha_{c}$, lo cual es consistente para los resultados obtenidos para $\alpha = 0.12$.

Por otro lado, al aumentar el parámetro de carga, la red disminuye su \textit{overlap} medio con los valores de entrada de manera continua. Esto es consecuencia de trabajar con sistemas finitos.

Para $\alpha > \alpha_{c}$, la distribución de los \textit{overlaps} no se concentra en $0$, si no que se acumulan alrededor de $m \approx 0.3$ a medida que aumenta $\alpha$. 