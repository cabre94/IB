\section*{Ejercicio 1 - Halcones y palomas extendidos}

\subsection*{Halcones, palomas y bravucones}

Queremos analizar el juego de halcones, palomas y bravucones mediante la dinámica del replicador, cuya matriz de payoff es
\begin{equation}
    A =
\begin{pmatrix}
\frac{G-C}{2} & G & G \\[4pt]
0 & \frac{G}{2} & 0 \\[4pt]
0 & G & \frac{G}{2}. \\
\end{pmatrix}
\end{equation}

Vamos a tomar $\vec{x} = \left(x,y,z\right)$ a las proporciones de halcones, palomas y bravucones, respectivamente. Para que la estrategia de halcón no sea dominante, tomaremos $C>G$. Ademas, recordamos la condición $x+y+z = 1$, con lo cual sera suficiente analizar solo dos ecuaciones de la dinámica. En particular, nosotros vamos a reemplazar $z=1-x-y$. Para la dinámica del replicador necesitamos los payoff de cada estrategia, ademas del payoff medio. Estos son
\begin{equation}
    f_x = \left(1,0,0\right) A \left(x,y,z\right)^t = \frac{G-C}{2}x + G\left(y+z\right) = G - \frac{G+C}{2}x
\end{equation}
\begin{equation}
    f_y = \left(0,1,0\right) A \left(x,y,z\right)^t = \frac{G}{2} y
\end{equation}
\begin{equation}
    f_z = \left(0,0,1\right) A \left(x,y,z\right)^t = Gy + \frac{G}{2}z = \frac{G}{2} \left(1-x+y\right)
\end{equation}
y para el payoff medio tenemos que
\begin{equation}
    \vec{f} = xf_x + yf_y + zf_z = -\frac{C}{2} x^2 + \frac{G}{2}.
\end{equation}

Con esto, ya podemos plantear las ecuaciones del replicador para $x$ e $y$
\begin{equation}
    \dot{x} = x \left(f_x - \vec{f}\right) = \cdots = \frac{x}{2} \left( Cx^2  - \left(G+C\right)x +G  \right)
    \label{eq1:x}
\end{equation}
\begin{equation}
    \dot{y} = y \left(f_y - \vec{f}\right) = \cdots = \frac{y}{2} \left(Cx^2 + Gy - G \right).
    \label{eq2:y}
\end{equation}

Pasamos ahora a analizar los puntos fijos del sistema. Si $x=0\Rightarrow \dot{x}=0$, luego para que $\dot{y} = 0$ tenemos dos opciones: una es que $y=0\rightarrow z = 1$, con lo cual $\left(0,0,1\right)$ es un punto fijo. si $y\neq 0$, entonces para el segundo termino de la Ec \ref{eq2:y} se anule, obtenemos que $y=1$, siendo entonces $\left(0,1,0\right)$ otro punto fijo.

Si $x\neq 0 $, el segundo termino de \ref{eq1:x} debe anularse. Vemos que
\begin{equation}
    Cx^2  - \left(G+C\right)x +G = C \left(x-1\right) \left(x \dfrac{G}{C}\right)
\end{equation}
con lo cual tenemos dos opciones. Si $x=1$, entonces como $x+y+z=1$, obtenemos $\left(1,0,0\right)$ como equilibrio. Si $x = \frac{G}{C}$, entonces para que $\dot{y} = 0$ necesitamos que $y=0$ con lo cual $\left(\frac{G}{C},0,1-\frac{G}{C}\right)$ es punto fijo o que $y = 1- \frac{G}{C}$, siendo entonces $\left(\frac{G}{C},1-\frac{G}{C},0\right)$ el ultimo punto fijo del sistema.

Para estudiar la estabilidad, calculamos las derivadas parciales de las ecuaciones \ref{eq1:x} y \ref{eq2:y} para obtener el Jacobiano, que resulta ser
\begin{equation}
    J =
\begin{pmatrix}
\frac{G}{2} - \left(G+C\right) x + \frac{3}{2}Cx^2  & 0 \\[6pt]
Cxy                                                 &  \frac{C}{2}x^2 + Gy - \frac{G}{2} \\
\end{pmatrix}
\end{equation}
con lo cual solo nos queda evaluar en los puntos fijos hallados.

\begin{equation}
J |_{\left(0,0,1\right)} =
\begin{pmatrix}
    \frac{C-G}{2}  & 0 \\[4pt]
    0       & \frac{C-G}{2}
\end{pmatrix}
\Rightarrow \text{Eq. inestable}
\end{equation}


\begin{equation}
J |_{\left(0,1,0\right)}=
\begin{pmatrix}
    \frac{G}{2}  & 0 \\[4pt]
    0       & \frac{G}{2}
\end{pmatrix}
\Rightarrow \text{Eq. inestable} 
\end{equation}


\begin{equation}
J |_{\left(1,0,0\right)} =
\begin{pmatrix}
    \frac{G}{2}  & 0 \\[4pt]
    0       & \frac{-G}{2}
\end{pmatrix}
\Rightarrow \text{Punto Saddle}
\end{equation}


\begin{equation}
J |_{\left(\frac{G}{C},0,1-\frac{G}{C}\right)} =
\begin{pmatrix}
    \frac{G}{2} \left( \frac{G}{C} -1 \right)  & 0 \\[4pt]
    0       & \frac{G}{2} \left( \frac{G}{C} -1 \right)
\end{pmatrix}
\Rightarrow \text{Eq. Estable}
\end{equation}


\begin{equation}
J |_{\left(\frac{G}{C},1-\frac{G}{C},0\right)} =
\begin{pmatrix}
    \frac{G}{2} \left( \frac{G}{C} -1 \right)  & 0 \\[4pt]
    0       & -\frac{G}{2} \left( \frac{G}{C} -1 \right)
\end{pmatrix}
\Rightarrow \text{Punto Saddle}
% \Rightarrow \text{Equilibrio }  ^{\text{Estables si } g<c}_{\text{Inestables si } g>c}
\end{equation}

En la Fig. \ref{ej1:Bravucones}, se muestra el simplex obtenido numéricamente, utilizando $C=3$ y $G=1$ 


\begin{figure}[htb!]
    \centering
    \includegraphics[width=0.8\textwidth]{../Bravucones.pdf}
    \caption{Simplex para el juego de halcones, palomas y bravucones.}
    \label{ej1:Bravucones}
\end{figure}



\subsection*{Halcones, palomas y vengadores}

Ahora pasamos a analizar el juego de halcones, palomas y vengadores mediante la dinámica del replicador, cuya matriz de payoff es
\begin{equation}
    A =
\begin{pmatrix}
\frac{G-C}{2} & G & \frac{G-C}{2} \\[6pt]
0 & \frac{G}{2} & \frac{G}{2} \\[6pt]
\frac{G-C}{2} & \frac{G}{2} & \frac{G}{2} \\
\end{pmatrix}
\end{equation}

Nuevamente tomamos $\vec{x} = \left(x,y,z\right)$, $C>G$ y reemplazamos $z = 1-x-y$. Los payoff en este caso son
\begin{equation}
    f_x = \left(1,0,0\right) A \left(x,y,z\right)^t = \frac{G-C}{2} \left(x+z\right) + Gy = \frac{G-C}{2} + \frac{G+C}{2}y
\end{equation}
\begin{equation}
    f_y = \left(0,1,0\right) A \left(x,y,z\right)^t = \frac{G}{2} \left(y+z\right) = \frac{G}{2} \left(1-x\right)
\end{equation}
\begin{equation}
    f_z = \left(0,0,1\right) A \left(x,y,z\right)^t = \frac{G-C}{2}x + \frac{G}{2} \left(y+z\right) = \frac{G}{2} - \frac{C}{2} x
\end{equation}
y para el payoff medio tenemos que
\begin{equation}
    \vec{f} = xf_x + yf_y + zf_z = \frac{C}{2} x^2 - Cx + Cxy + \frac{G}{2}.
\end{equation}

Y las ecuaciones de la dinámica del replicador resultan
\begin{equation}
    \dot{x} = x \left(f_x - \vec{f}\right) = \cdots = x \left( -\frac{C}{2}x^2 - Cxy + Cx + \frac{G+C}{2}y -\frac{C}{2} \right)
    \label{eq2:x}
\end{equation}
\begin{equation}
    \dot{y} = y \left(f_y - \vec{f}\right) = \cdots = yx \left( -\frac{G}{2} + C - \frac{C}{2} x - Cy \right).
    \label{eq22:y}
\end{equation}

Buscando los puntos fijos (vamos a saltear algunas cuentas sin mucho interés) obtenemos tres puntos fijos. $\left(0, y, 1-y\right)$, $\left(1,0,0\right)$ y $\left(\frac{G}{C}, 1-\frac{G}{C}, 0\right)$. En particular, $\left(0, y, 1-y\right)$ es punto fijo par cualquier valor de $y$, con lo cual tenemos infinitos puntos fijos ubicados en uno de los lados del simplex.

Calculamos las derivadas parciales de las ecuaciones \ref{eq2:x} y \ref{eq22:y} para obtener el Jacobiano, que resulta ser
\begin{equation}
    J =
\begin{pmatrix}
-\frac{3}{2}Cx^2 - 2Cxy + 2Cx + \frac{G+C}{2}y - \frac{C}{2} & -Cx^2 + \frac{G+C}{2}x \\[6pt]
-\frac{G}{2}y + Cy - Cxy - C y^2  &  -\frac{G}{2}x + Cx - \frac{C}{2} x^2 - 2Cxy \\
\end{pmatrix}
\end{equation}
y evaluamos en los puntos fijos hallados
\begin{equation}
J |_{\left(1,0,0\right)} =
\begin{pmatrix}
    0  & -C + \frac{G+C}{2} \\[4pt]
    0  & \frac{C-G}{2}
\end{pmatrix}
\Rightarrow \text{Como $C > G$, tenemos una acumulación de inestables.}
\end{equation}


\begin{equation}
J |_{\left(\frac{G}{C},1-\frac{G}{C},0\right)}=
\begin{pmatrix}
    0  & -\frac{1}{2}\frac{G^2}{C} + \frac{G}{2} \\[6pt]
    \frac{1}{2}\frac{G^2}{C} - \frac{G}{2}       & \frac{G^2}{C} - G
\end{pmatrix}
% \Rightarrow \text{Eq. inestable} 
\end{equation}
en este caso, los autovalores son ambos iguales a $\frac{G}{C}\frac{G-C}{2}$. Como consideramos $C >G$, tenemos que este punto de equilibrio es estable. Por ultimo,
\begin{equation}
J |_{\left(0,y,1-y\right)} =
\begin{pmatrix}
    \frac{G+C}{2}y - \frac{C}{2}  & 0 \\[6pt]
    y\left(-\frac{G}{2} + C - Cy\right)       & 0
\end{pmatrix}
% \Rightarrow \text{Punto Saddle}
\end{equation}
en donde dependiendo de si $\frac{G+C}{2}y - \frac{C}{2}$ es mayor o menor que 0, tendremos una acumulación de inestables o estables respectivamente. Es decir que el lado del simplex donde tenemos infinitos puntos de equilibrio, un segmento sera acumulación de estables y el restante sera de inestables.

En la Fig. \ref{ej1:Vengadores}, se muestra el simplex obtenido numéricamente, utilizando $C=3$ y $G=1$ 

\begin{figure}[ht!]
    \centering
    \includegraphics[width=0.8\textwidth]{../Vengativos.pdf}
    \caption{Simplex para el juego de halcones, palomas y vengativos.}
    \label{ej1:Vengadores}
\end{figure}