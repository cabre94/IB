\section*{Ejercicio 6}
\graphicspath{{Figuras/ej_06/}}

Tenemos el modelo para describir el efecto Allee
\begin{equation}
    \frac{dN}{dt} = f(N) = r N \left[ 1 - \frac{N}{K} \right] \left[ \frac{N}{A} -1 \right]
\end{equation}
de donde planteando que $f(N) = 0$ obtenemos que los puntos fijos son 0, $A$ y $K$. Para estudiar la estabilidad calculamos la derivada $f'(N)$ obteniendo
\begin{equation}
    f'(N) = r \left[ 1-\frac{N}{K} \right] \left[ \frac{N}{A} -1 \right] + \frac{rN}{A} \left[ 1 - \frac{N}{K} \right] - \frac{rN}{K} \left[ \frac{N}{A}-1\right]
\end{equation}
que evaluando en los puntos fijos obtenidos, tenemos que
\begin{equation}
    f'(N=0) = -r; \indent f'(N=A)=r\left[1-\frac{A}{K}\right]; \indent f'(N=K)=r\left[1-\frac{K}{A}\right]. 
\end{equation}

Para seguir avanzando con el análisis, tomamos el caso en que $r>0$ y $0<A<K$. En esta situación, los puntos fijos en $0$ y $K$ son puntos fijos estables, mientras que el punto fijo en $A$ es inestable. En la Figura \ref{06_Simulacion} se observa la simulación obtenido del sistema para distintas condiciones iniciales. En esta figura se observan dos regímenes, uno en el intervalo $[0,A)$ en donde el sistema evoluciona al equilibrio estable en $0$ y la población se extingue y otro en $[A,\infty)$ en donde el sistema evoluciona al equilibrio estable en $K$ y la población sobrevive. En la Figura \ref{06_Funcion_y_raices} se observa la función que determina la evolución del sistema indicando con flechas sobre el eje de abscisa la evolución del sistema.

\begin{figure}[hb!]
    \centering
    \begin{subfigure}[b]{0.49\textwidth}
        \includegraphics[width=\textwidth]{Simulacion.pdf}
        \caption{}
        \label{06_Simulacion}
    \end{subfigure}
    % \hfill
    \begin{subfigure}[b]{0.49\textwidth}
        \includegraphics[width=\textwidth]{f.pdf}
        \caption{}
        \label{06_Funcion_y_raices}
    \end{subfigure}
    \caption{En la Figura (a) se observa la evolución del sistema para distintas condiciones iniciales, tomando $r>0$ y $0<A<K$. Se observan dos regímenes, uno en el intervalo $[0,A)$ en donde el sistema evoluciona al equilibrio estable en $0$ y la población se extingue y otro en $[A,\infty)$ en donde el sistema evoluciona al equilibrio estable en $K$ y la población sobrevive. En la Figura (b) se observa la función que determina la evolución del sistema indicando con flechas sobre el eje de abscisa la evolución del sistema. }
    \label{06_ejercicio}
\end{figure}

La diferencia que puede observarse respecto a la ecuación logística, ademas de la aparición de un nuevo punto fijo en $A$, es que la aparición de este nuevo punto fijo genera un cambio en la estabilidad del punto fijo en $0$, el cual es inestable en la ecuación logística mientras que en este caso dicho punto es estable (para $r>0$). De esta forma, como ya se remarco previamente, existe un numero mínimo ($A$) para la población para que la supervivencia sea posible.

En caso de que $0<K<A$, las estabilidades de los puntos fijos en $A$ y $K$ cambian, pero también cambian sus posiciones, con lo cual la situación es análoga a cuando $0<A<K$.
%Por otro lado, si $r<0$, la estabilidad de todos los puntos fijos se invierte